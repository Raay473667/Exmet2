\documentclass[a4paper,12pt]{article}
\usepackage [utf8x]{inputenc}
\usepackage[czech]{babel} 
\usepackage{graphicx}
\usepackage{amsmath}
\usepackage{xspace}
\usepackage{url}
\usepackage{indentfirst}
\usepackage[margin=22mm]{geometry}
\usepackage{esvect}
\usepackage{ragged2e}
\usepackage{tikz,pgf}
\usepackage{bm}
\usepackage{perpage}
\usepackage{capt-of}
\usepackage{hyperref}

\hypersetup{
	colorlinks=true,
	filecolor=magenta,      
	urlcolor=cyan,
}

\graphicspath{
	{img/}
	{plots/}
}

\MakeSorted{figure}
\newtoks\jmenopraktika \newtoks\jmeno \newtoks\datum
\newtoks\obor \newtoks\skupina \newtoks\rocnik \newtoks\semestr
\newtoks\cisloulohy \newtoks\jmenoulohy
\newtoks\tlak \newtoks\teplota \newtoks\vlhkost
\jmenopraktika={Rentgenová difrakce}  
% nahradte jmenem vaseho predmetu
\jmeno={Radek Horňák}            
\datum={23.~února 2022}        % nahradte datem mereni ulohy
\obor={F}                     
\skupina={Středa 10:00}            
\rocnik={2.}                  
\semestr={IV.}                 
\cisloulohy={6}    % cislo ulohy           

\begin{document}
	\begin{center}
		{\Large Přírodovědecká fakulta Masarykovy univerzity} \\
		\bigskip
		{\Large \bfseries EXPERIMENTÁLNÍ METODY 2} \\
		\bigskip
		{\Large \the\jmenopraktika}
	\end{center}
	\bigskip
	\noindent
	\setlength{\arrayrulewidth}{1pt}
	\begin{tabular*}{\textwidth}{@{\extracolsep{\fill}} l l}
		\large {\bfseries Zpracoval:}  \the\jmeno  \\ \\ \large  
		{\bfseries Skupina:} \the\skupina \hspace{40mm} \large  
		{\bfseries Datum:} \the\datum\\ \\
		\hline
	\end{tabular*}
	
	\section{Úvod}
\par Rentgenová difrakce, pro kterou používáme zkratku XRD (z angl. X-ray 
diffraction) je analytická metoda pro určení atomové a molekulární 
struktury krystalu. Krystalová mřížka způsobí difrakci dopadajícího 
rentgenového záření. Měřením úhlů a intenzit rozptýleného záření lze určit 
střední polohy atomů v krystalu, chemické vazby, rozměr elementární buňky aj.
\par První Braggovou difrakční podmínkou je dopad monochromatického zážení na 
polykrystalický vzorek, jehož zrna jsou tak malá, že v ozářeném objemu jsou 
zastoupeny všechny orientace zrn. Tyto zrna difraktují dle rovnice
\begin{equation}
	2a \sin \theta = \lambda \sqrt{(h^2+k^2+l^2)}  
\end{equation}
kde $\lambda$ je vlnová délka záření, $a$ je mřížkový parametr a $h,k,l$ jsou 
Laueho indexy. Ty vzniknou vynásobením Millerových indexů $h_0,k_0,l_0$ roviny 
přirozeným číslem $n$, které je řádem difrakce.
\par Intenzita difrakce závisí na struktuře elementární buňky. Pro prostou 
kubickou mřížku jsou všechny difrakce povoleny. Prostorově centrovaná kubická 
mřížka má povolené difrakce, jejichž součet Laueho indexů je sudý. V plošně 
centrované kubické mřížce jsou povoleny difrakce, pokud Laueho indexy jsou buď 
všechny sudé, nebo všechny liché. Diamantová mřížka má povolené difrakce, pokud 
jsou Laueho indexy liché a nebo jsou sudé a zároveň je jejich součet dělitelný 
čtyřmi.
\par Pro výpočet velikosti krystalitů se využívá Scherrerovy rovnice:

\begin{equation}
	\tau = \dfrac{K\lambda}{\beta cos\Theta}
\end{equation}
$\tau$ je velikost krystalitů, $K$ je empirický faktor udávající tvar 
krystalitů (typicky 0,94), $\beta$ je pološířka píku v polovině jeho maxima a 
$\theta$ je úhel udávající pozici píku.

\section{Měření}
\par Měření se provádí v Bragg-Brentanově konfiguraci -- samofokusující 
konfiguraci. Před samotným měřením je potřeba přístroj najustovat, tedy 
nastavit 
vzádelnosti zdroj-střed goniometru a střed goniometru-detektor na stejnou 
hodnotu, a také aby úhel dopadu středního paprsku byl roven jeho úhlu výstupu 
ze vzorku do detektoru. Zdrojem záření je Cu rentgenka obsahující vlnové délky 
K$_{\alpha1}$ a K$_{\alpha2}$, tedy 1,540601\,\AA\ respektive 1,544430\,\AA. 
Sollerovy clony vymezují úhlovou aperturu ve směru kolmém na rozptylovou rovinu 
a motorizovaná štěrbina ve směru podélném.
\par Měřenou látkou je kuchyňská sůl. Při měření je využito měřicího rozsahu 
$20 -- 100$\,°~s krokem 0,01\,°. Výstupem je závislost měřené intenzity na 
difrakčním úhlu $2\theta$.




	
	\clearpage
	\section{Závěr}

	
	
\end{document}
