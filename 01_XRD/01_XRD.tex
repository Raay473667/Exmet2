\documentclass[a4paper,12pt]{article}
\usepackage [utf8x]{inputenc}
\usepackage[czech]{babel} 
\usepackage{graphicx}
\usepackage{amsmath}
\usepackage{xspace}
\usepackage{url}
\usepackage{indentfirst}
\usepackage[margin=22mm]{geometry}
\usepackage{esvect}
\usepackage{ragged2e}
\usepackage{tikz,pgf}
\usepackage{bm}
\usepackage{perpage}
\usepackage{capt-of}
\usepackage{hyperref}
\usepackage{siunitx}
\usepackage{booktabs}

\hypersetup{
	colorlinks=true,
	filecolor=magenta,      
	urlcolor=cyan,
}

\graphicspath{
	{img/}
	{plots/}
}

\MakeSorted{figure}
\newtoks\jmenopraktika \newtoks\jmeno \newtoks\datum
\newtoks\obor \newtoks\skupina \newtoks\rocnik \newtoks\semestr
\newtoks\cisloulohy \newtoks\jmenoulohy
\newtoks\tlak \newtoks\teplota \newtoks\vlhkost
\jmenopraktika={Rentgenová difrakce}  
% nahradte jmenem vaseho predmetu
\jmeno={Radek Horňák}            
\datum={28.~února 2023}        % nahradte datem mereni ulohy
\obor={F}                     
\skupina={Středa 10:00}            
\rocnik={2.}                  
\semestr={IV.}                 
\cisloulohy={6}    % cislo ulohy           

\begin{document}
	\begin{center}
		{\Large Přírodovědecká fakulta Masarykovy univerzity} \\
		\bigskip
		{\Large \bfseries EXPERIMENTÁLNÍ METODY 2} \\
		\bigskip
		{\Large \the\jmenopraktika}
	\end{center}
	\bigskip
	\noindent
	\setlength{\arrayrulewidth}{1pt}
	\begin{tabular*}{\textwidth}{@{\extracolsep{\fill}} l l}
		\large {\bfseries Zpracoval:}  \the\jmeno  \hspace{50mm} \large  
		{\bfseries Datum:} \the\datum\\ 
		\hline
	\end{tabular*}
	
\section{Úvod}\noindent
Rentgenová difrakce, pro kterou používáme zkratku XRD (z angl. X-ray 
diffraction) je analytická metoda pro určení atomové a molekulární 
struktury krystalu. Krystalová mřížka způsobí difrakci dopadajícího 
rentgenového záření. Měřením úhlů a intenzit rozptýleného záření lze určit 
střední polohy atomů v krystalu, chemické vazby, rozměr elementární buňky aj.

První Braggovou difrakční podmínkou je dopad monochromatického záření na 
polykrystalický vzorek, jehož zrna jsou tak malá, že v ozářeném objemu jsou 
zastoupeny všechny orientace zrn. Tyto zrna difraktují dle Braggovy rovnice, 
která má pro kubickou mřížku tvar:
\begin{equation}
	2a \sin \theta = \lambda \sqrt{(h^2+k^2+l^2)}  
	\label{eq:parameretr}
\end{equation}
kde $\lambda$ je vlnová délka záření, $a$ je mřížkový parametr a $h,k,l$ jsou 
Laueho indexy. Ty vzniknou vynásobením Millerových indexů $h_0,k_0,l_0$ roviny 
přirozeným číslem $n$, které je řádem difrakce.

Intenzita difrakce závisí na struktuře elementární buňky. Pro prostou 
kubickou mřížku jsou všechny difrakce povoleny. Prostorově centrovaná kubická 
mřížka má povolené difrakce, jejichž součet Laueho indexů je sudý. V plošně 
centrované kubické mřížce jsou povoleny difrakce, pokud Laueho indexy jsou buď 
všechny sudé, nebo všechny liché. Diamantová mřížka má povolené difrakce, pokud 
jsou Laueho indexy liché a nebo jsou sudé a zároveň je jejich součet dělitelný 
čtyřmi.
\par Pro výpočet velikosti krystalitů se využívá Scherrerovy rovnice:

\begin{equation}
	\tau = \dfrac{K\lambda}{\beta cos\theta}
	\label{eq:krystaly}
\end{equation}
$\tau$ je velikost krystalitů, $K$ je empirický faktor udávající tvar 
krystalitů (typicky 0,94), $\beta$ je pološířka píku v polovině jeho maxima a 
$\theta$ je úhel udávající pozici píku.

\section{Měření}\noindent
Měření se provádí v Bragg-Brentanově konfiguraci -- samofokusující 
konfiguraci. Před samotným měřením je potřeba přístroj najustovat, tedy 
nastavit 
vzdálenosti zdroj-střed goniometru a střed goniometru-detektor na stejnou 
hodnotu, a také aby úhel dopadu středního paprsku byl roven jeho úhlu výstupu 
ze vzorku do detektoru. Zdrojem záření je Cu rentgenka obsahující vlnové délky 
K$_{\alpha1}$ a K$_{\alpha2}$, tedy 1,540601\,\AA\ respektive 1,544430\,\AA. 
Jsou zde také přítomné čáry K$_{\beta}$ a z wolframové žhavené 
katody W$_\alpha$. Tyto čáry jsou potlačeny niklovým filtrem. 
Sollerovy clony vymezují úhlovou aperturu ve směru kolmém na rozptylovou rovinu 
a motorizovaná štěrbina ve směru podélném.

Měřenou látkou je měď. Při měření je využito měřicího rozsahu 
$20 - 100$°~s krokem 0,01°. Výstupem je závislost měřené intenzity na 
difrakčním úhlu $2\theta$.

\section{Zpracování měření}\noindent
Na obr.~\ref{fig:spectrum} vidíme naměřený difraktogram s odečteným 
pozadím -- odečetl jsem konstantní hodnotu 5300. Pozorujeme, že Laueho indexy 
jsou všechny buď sudé nebo liché, což znamená, že se jedná o plošně centrovanou 
kubickou mřížku FCC. Následně jsem na obr.~\ref{fig:43} až \ref{fig:95} fitoval 
jednotlivé 
píky. Jelikož je každý pík rozštěpený na dva, což odpovídá rozdílné vlnové 
délce K$_{\alpha1}$ a K$_{\alpha2}$, fitoval jsem součtem dvou Lorentzových 
funkcí, které odpovídaly měřeným datům lépe, než funkce Gaussovy.

Do tabulky~\ref{tab:fwhm} jsem vynesl polohy maxim píků K$_{\alpha1}$ 
a~K$_{\alpha2}$, roviny difrakce jim příslušné a~pološířku (FWHM) fitovaných 
píků.
Následně jsem dle rovnice~\eqref{eq:parameretr} určil mřížkový parametr pomocí 
obou čar K$_{\alpha1}$ a K$_{\alpha2}$. Výsledný mřížkový parametr určený jako 
průměr hodnot je $$a = (3,623\pm0,002)\,\si{\angstrom}$$
Z pološířky píků jsem 
dle rovnice~\eqref{eq:krystaly} spočítal velikost krystalitů. Průměrná velikost 
krystalitů je $$\tau = (60 \pm 2)\,\si{\nano\meter}$$ 

\begin{center}
	\begin{table}[h!]
		\centering
		\caption{\centering Polohy maxim píků, roviny difrakcí, FWHM, mřížkový 
		parametr a velikost krystalitů daných píků.}
		\label{tab:fwhm}
		\begin{tabular}{c|c|c|c|c|c} \toprule
			Poloha maxima K$_{\alpha1}$ $2\theta$\,[\si{\degree}] & 43,13 & 
			50,28 & 73,98 & 
			89,80 & 95,02 
			\\
			Poloha maxima K$_{\alpha2}$ $2\theta$\,[\si{\degree}] & 43,24 &	
			50,40 &	74,19 &	90,08 &	95,33 \\\midrule
			Rovina difrakce & (111) & (200) & (220) & (311) & (222) \\
			FWHM\,[\si{\degree}] & 0,1410 & 0,1569 & 0,1642 & 0,2216 & 0,1958 
			\\ \midrule
			Mřížkový parametr $a_1$\,[\si{\angstrom}] & 3,6298&	3,6267&	3,6211&	
			3,6193&	3,6188 \\
			Mřížkový parametr $a_2$\,[\si{\angstrom}] & 3,6300&	3,6271&	3,6212&	
			3,6194&	3,6187 \\
			Velikost krystalitů $\tau$\,[\si{\nano\meter}] & 63,27 & 58,43 & 
			63,26 & 52,86 & 62,73 \\
				
		\bottomrule
		\end{tabular}
	\end{table}
\end{center}

\begin{figure}[h!]
	\centering
	%\vspace*{-8mm}
	\includegraphics[width=\textwidth]{xrdplot.pdf}
	%\vspace*{-2mm}
	\caption{\centering Difraktogram s rozpoznanými píky.}
	\label{fig:spectrum}
\end{figure}

\begin{figure}[h!]
	\centering
	%\vspace*{-8mm}
	\includegraphics[width=\textwidth]{xrdplot43.14.pdf}
	%\vspace*{-2mm}
	\caption{\centering Fit píku (111).}
	\label{fig:43}
\end{figure}

\begin{figure}[h!]
\centering
%\vspace*{-8mm}
\includegraphics[width=\textwidth]{xrdplot50.28.pdf}
%\vspace*{-2mm}
\caption{\centering Fit píku (200).}
\label{fig:50}
\end{figure}

\begin{figure}[h!]
\centering
%\vspace*{-8mm}
\includegraphics[width=\textwidth]{xrdplot74.pdf}
%\vspace*{-2mm}
\caption{\centering Fit píku (220).}
\label{fig:74}
\end{figure}

\begin{figure}[h!]
\centering
%\vspace*{-8mm}
\includegraphics[width=\textwidth]{xrdplot89.8.pdf}
%\vspace*{-2mm}
\caption{\centering Fit píku (311).}
\label{fig:90}
\end{figure}
	
\begin{figure}[h!]
	\centering
	%\vspace*{-8mm}
	\includegraphics[width=\textwidth]{xrdplot95.0.pdf}
	%\vspace*{-2mm}
	\caption{\centering Fit píku (222).}
	\label{fig:95}
\end{figure}
\clearpage
\section{Závěr}\noindent
V této úloze jsme se seznámili s rentgenovou difrakcí, metodou pro určení 
atomové a molekulární struktury krystalu. Měřili jsme vzorek mědi. U pěti 
nejvýznamnějších píků jsme určili jejich rovinu difrakce. Z nich můžeme říct, 
že mřížka je plošně centrovaná kubická FCC. Z~polohy maxim jsme vypočítali 
mřížkový parametr $$a = (3,623\pm0,002)\,\si{\angstrom}$$ což odpovídá 
tabulkovým hodnotám mřížkového parametru mědi v~rozsahu teplot 
$293\,\si{\kelvin}$ 
a~$577\,\si{\kelvin}$~\cite{cod}:
%což odpovídá rozmezí teplot tabulkových hodnot 
$$a_{\text{293\,\si{\kelvin}}} = 3,613\,\si{\angstrom}$$
$$a_{\text{577\,\si{\kelvin}}} = 3,630\,\si{\angstrom}$$ 

Dále jsme určili fitováním Lorentzových funkcí pološířky píků. Z tohoto údaje 
jsme vypočítali velikost krystalitů $$\tau = (60 \pm 2)\,\si{\nano\meter}$$

V rámci mé diplomové práce se věnuji osciloskopické a spektroskopické 
diagnostice plaz\-mo\-vé\-ho výboje uvnitř kavitačního mraku. Nemám tedy žádný 
materiál, u něhož bych mohl využít XRD metodu.
	
\begin{thebibliography}{10}
	%\bibitem{copper} Simon, N.J., Drexler, E., \& Reed, R.P. (1992). 
	%\textit{Properties of copper and copper alloys at cryogenic temperatures. 
	%Final report.}
	\bibitem{cod} Suh, IK., Ohta, H. \& Waseda, Y. \textit{High-temperature 
	thermal expansion of six metallic elements measured by dilatation method 
	and X-ray diffraction.} J Mater Sci 23, 757–760 (1988). 
	https://doi.org/10.1007/BF01174717
\end{thebibliography}
	
\end{document}
